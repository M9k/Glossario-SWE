\documentclass{article}
\usepackage{indentfirst}
\usepackage[utf8]{inputenc}
\usepackage[margin=0.5in]{geometry}
\usepackage{graphicx}
\usepackage{verbatim}

\begin{document}

\title{Glossario di SWE}
\author{M9k}
\maketitle

%\section{TEXT}
%\subsection{TEXT}
%\subsubsection{TEXT}
%\paragraph{TEXT}
%\subparagraph{Subparagraph}Subparagraph text.\vspace{2mm}
	
\section{Glossario}
		\textbf{Progetto}\\
		Insieme di attività e compiti\\
		-per raggiungere obbiettivi con specifiche fissate\\
		-data di inizio e di fine fissate\\
		-risorse limitate (es: persone, tempo, fondi, strumenti)\\
		-consuma risorse svolgendosi\\
			
		\textbf{Processo}\\
		\textit{Insieme di attività correlate} e \textit{coese} che trasformano ingressi (bisogni) in uscite (prodotti) secondo regole date, consumando risorse nel farlo\\
		Correlate: hanno un motivo/una capacità per stare assieme\\
		Coese: utili al medesimo obiettivo\\
		
		\textbf{Attività}\\
		Cosa da fare, che voglio fare, per il raggiungimento degli obbiettivi, composta da più compiti\\
		
		\textbf{Compito}\\
		Cosa che una persona deve fare, che va fatta\\

		\textbf{Fasi principali:}\\
		-Pianificazione (\textit{gestione risorse} e \textit{responsabilità})\\
		-Analisi dei requisiti (\textsl{cosa} devo fare)\\
		-Progettazione (\textit{come} farlo)\\
		-Realizzazione (con una \textit{qualità}, \textit{verificando} la correttezza, \textit{validando} i risultati)\\
		
		\textbf{Efficienza}\\
		Produttività, metrica del grado di riduzione degli sprechi\\
		Quantità prodotto realizzato/risorse utilizzate\\
		
		\textbf{Efficacia}\\
		Qualità, metrica del grado di raggiungimento degli obbiettivi interni (del fornitore) o esterni (gradimento del cliente)\\
		
		\textbf{Iterazione}\\
		Può essere anche un incremento, procedere per raffinamento o rivisitazioni (pittura)\\
		Non so se sto migliorando o meno, non quantificabile, non efficiente, rifinisco gli aspetti senza magari avanzare, non so a che punto sono\\
		
		\textbf{Incremento}\\
		Procedere per aggiunta a un impianto base (scultura)\\
		Si progredisce a punti, a baseline, quantificabile\\
		
		\textbf{Prototipo}\\
		Per provare e capire meglio, usa e getta (bozza), oppure per avere avanzamento incrementale (baseline)\\
					
		\textbf{Baseline}\\
		Prodotto prototipale, è il risultato di avanzamenti misurabili\\
		
		\textbf{Milestone}\\
		Concretizzata da almeno una baseline, punto nel tempo strategico e di riferimento, meta da raggiungere con dei risultati certi/solidi che non deve essere ritoccata (incrementali)\\
		
		\textbf{Prodotto SW}\\
		È un insieme di parti, che stanno assieme secondo la loro \textit{configurazione}.\\
		Ogni sistema fatto di parti va gestito con il \textit{controllo di configurazione}.\\
			
		\textbf{Configurazione}\\
		Modo nel quale si assemblano i pezzi di un software (ordine, parti, librerie, impostazioni, etc)\\
		Usato per il build, si gestisce con il controllo di configurazione\\
			
		\textbf{Metrica}\\
		Metodo di misurazione, l'unità di misura da sola è insignificante\\
		
		\textbf{Requisiti}\\
		1 - Condizione (capability) da chi la offre - capacità di risolvere un problema o raggiungere un obbiettivo\\
		2 - Condizione (capability) da chi la richiede - che deve essere soddisfatta o posseduta da un sistema per aderire a un obbligo (contratto, standard, specifica, documento formale)\\
		3 - Descrizione documentata di una condizione come in 1 o 2\\
		
		\textbf{Qualifica}\\
		Verifica + Validazione\\
		\textit{Verifica}: processo di supporto, accertamento che l'esecuzione delle attività non abbia introdotto errori, rivolto ai processi, da fare per OGNI componente\\
		Per la verifica serve piano di verifica che si basa sul way of working\\
		\textit{Validazione}: controllo rivolto SOLO al prodotto finale, lungo e costoso, accertarsi che il prodotto realizzato corrisponda alle attese\\
		
		
		
		
		
		
		
		
		
		
		
		
		
		
		
		
		
			
	\clearpage
	\section{Ingegneria}
		\textbf{Ingegneria}\\
			Applicazioni principi matematici e scientifici a scopo pratico, NON per esplorare nuove possibilità o espandere la scienza\\
			Mai inventare, utilizzare sempre metodi testati e funzionanti\\
			
		\textbf{Best practice}\\
		Miglior modo (way of working) per raggiungere uno scopo, secondo applicazioni passate che hanno dimostrato i risultati\\
		
		\textbf{Pratical ends}\\
		Avere un fine civile e sociale oltre che economico\\
		
	\subsection{Ingegneria del software}
		\textbf{Ingegneria del software}\\
		Disciplina per la realizzazione di \textit{prodotti software} impegnativo e che richiede collaborazione\\
		-in grande e in piccolo (tanto in quantità o poco e specializzato)\\
		-con qualità = \textit{efficacia} = grado di conformità, capacità di raggiungere gli obiettivi\\
		-con costi e tempi contenuti = \textit{efficienza} = capacità di ridurre le risorse e gli sprechi, seguendo la best practice \\
		-tutto lungo il \textit{ciclo di vita}\\
		
		\textbf{Ingegneria del software}\\
		Raccogliere, organizzare e consolidare conoscenza (body of knowledge) necessarie a realizzare progetti SW con massima efficacia e efficenza.\\
		Acquisire, utilizzare e mantenere i best practice.\\
		
		
		\textbf{Ingegneria del software}\\
		Secondo IEEE: Approccio \textit{sistematico}, \textit{disciplinato} e \textit{quantificato} allo sviluppo, uso, manutenzione e ritiro del SW.\\
		Sistematico: metodico e rigoroso, usando una metodologia precisa, per studiare ed evolvere best practice\\
		Disciplinato: regole fissate\\
		Quantificabile: efficienza ed efficacia misurabili.\\
		
		
		\textbf{Tipologie di prodotti software}\\
		-Commessa: forma, contenuto e funzioni definiti dal committente\\
		-Pacchetto: forma, contenuto e funzioni idonei alla replicazione\\
		-Componente: forma, contenuto e funzioni idonei alla composizione\\
		-Servizio: forma, contenuto e funzioni definiti dal problema\\
		
		\textbf{Le 4 P di SWE}\\
		-People (stakeholder e team di sviluppo)\\
		-Product (SW e documentazione)\\
		-Project (Insieme di attività di produzione)\\
		-Process (way of working)\\
		
		\textbf{Ciclo di vita}\\
		Insieme di stati di avanzamento del software fino al ritiro\\
		Un ciclo di vita lungo porta a elevati costi di \textit{manutenzione}\\
		
		\textbf{Manutenzione}\\
		-correttiva: fix dei bug\\
		-adattiva: rifinisco i requisiti\\
		-evolutiva: evoluzione del software secondo i nuovi usi\\
		
		\textbf{Utilità}\\
		\textit{Metrica} riguardante gli utilizzi/utenti di un prodotto nel tempo\\
		
		\includegraphics[width=14cm]{semat_org.jpg}\\
		
		
	\clearpage
	\section{Processi SW}
		\textbf{Ciclo di vita}\\
		Gli stati che il prodotto assume dal concepimento al ritiro\\
		Serve per valutare costi, tempi, obblighi e rischi PRIMA di svolgere il progetto\\
		Scelta tra più possibili cicli di vita, ognuno con vantaggi e limiti\\
		
		\textbf{Processi di ciclo di vita}\\
		Specificano le attività da svolgere per abilitare corrette transizioni di stato nel ciclo di vita\\
		
		
		\textbf{Modelli di ciclo di vita}\\
		Descrivono come i processi di ciclo di vita si relazionano tra di loro rispetto agli stati\\
		Aiutano a pianificare, organizzare ed eseguire lo svolgimento delle attività\\
		Svariati, scelgo in base alla situazione, ognuno con pregi e limiti\\
		
		\textbf{Ciclo di sviluppo}\\
		Ciclo di vita fino alla consegna, senza utilizzo, manutenzione e ritiro\\
		
		\textbf{Visione a grafi}\\
		Gli stati sono i nodi (concezione, sviluppo, utilizzo, ritiro, etc), gli archi le attività svolte sul prodotto necessarie per farlo avanzare.\\
		Natura degli stati e pre- e post- condizione determinate da \textit{obblighi} (vincoli contrattuali), \textit{regole} (standard di processo) e \textit{strategie}\\
		
		\textbf{Modelli più significativi}\\
		-Sequenziale o a cascata (waterfall)\\
		-Incrementale\\
		-A evoluzioni successive\\
		-A spirale\\
		-Per componenti\\
		-Agile\\
		
		\textbf{Riuso}\\
		-Occasionale: copia-incolla, basso costo, scarso impatto, da evitare\\
		-Sistematico: per progetto/prodotto/azienda, maggior costo, maggior impatto\\
		
		\textbf{Malleabilità}\\
		Un buon software non è statico, ma si modifica e si addatta in quanto usandolo si scoprono migliorie e/o cambiano gli usi\\
			
		\textbf{Processo}\\
		\textit{Insieme di attività correlate} e \textit{coese} che trasformano ingressi (bisogni) in uscite (prodotti) secondo regole date, consumando risorse nel farlo\\
		Correlate: sono collegate, hanno la capacità di stare assieme\\
		Coese: hanno un motivo di stare assieme\\
		\includegraphics[width=14cm]{processi.jpg}\\
		Risorse: efficienza = produttività, cosa ho fatto/quante risorse ho utilizzato\\
		Misurazione: efficacia, raggiungimento di obbiettivi interni (del fornitore, cioè di chi crea il software) o esterni (gradimento da parte del cliente)\\
		
		\textbf{Economicità}\\
		Insieme di efficienza ed efficacia, da controllare DURANTE lo sviluppo usando:\\
		-dati tempestivi (non si può attendere la fine, sarebbe troppo tardi)\\
		-dati accurati (niente opinioni personali ma numeri)\\
		-non intrusività (non bloccare il lavoro per controllare il progresso)\\

		\textbf{Standard di processo}\\
		Voluti dai committenti per vincolare il fornitore\\
		Per facilitare \textit{controllo}, \textit{collaudo} e \textit{accettazione}\\
		Settoriali o generali/trasversali\\
		Vincolo (imposto) o riferimento (non imposto, come modello)\\
		
		\textbf{Standard come modello di azione}\\
		Sono una serie di passaggi da compiere, guida passo a passo, come una ricetta\\
		Definizione e imposizione di \textit{procedure}, definizione e proposizione di \textit{processi da specializzare}\\
		
		\textbf{Standard come modello di valutazione}\\
		Servono per avere una valutazione sul comportamento del progetto\\
		Modelli più generali, copre più contesti, per identificare best practice\\
	
		\textbf{ISO/IEC 12207:1995}\\
		Letta come 12 207\\
		Più diffuso, ad alto livello, molto astratto, preso spunto dagli standard militari del dipartimento di difesa\\
		Identifica i processi di ciclo di vita del SW\\
		Struttura modulare che richiede specializzazione\\
		Specifica le responsabilità sui processi e i prodotti\\
		Tre parti principali: processi primari, di supporto e organizzativi\\
	
		\textbf{Processi primari}\\
		Necessari per l'esistenza di un progetto\\
		ES:\\
		-Fornitura (gestione rapporti con il cliente, primo passo di un progetto)\\
		-Acquisizione (gestione dei sotto-fornitori)\\
		-Sviluppo\\
		-Gestione operativa (utilizzo, erogazione, installazione)\\
		-Manutenzione (correzione, adattamento, evoluzione)\\
		
		\textbf{Processi di supporto}\\
		ES:\\
		-Documentazione\\
		-Accertamento qualità\\
		-Gestione delle versioni e delle configurazioni\\
		-Qualifica: verifica + validazione\\
		-Revisioni congiunte con il cliente\\
		-Verifiche ispettive interne\\
		-Risoluzione dei problemi (gestione dei cambiamenti)\\
		
		\textbf{Processi organizzativi}\\
		ES:\\
		-Gestione dei processi\\
		-Gestione delle infrastrutture\\
		-Miglioramento del processo\\
		-Formazione personale\\
		
		\textbf{Tecniche}\\
		Ricette per svolgere determinati compiti\\
		Vincoli o strategie restringono il grado di libertà\\
		
		\textbf{Buona organizzazione}\\
		Si basa sul riconoscere i processi, adottarli consapevolmente ed efficacemente e supportarli in modo efficiente\\
		
		\begin{comment}
			\textbf{Tipi di processo}\\
			-Standard: di base, generico, condiviso tra aziende nello stesso dominio applicativo\\
			-Definito: specializzazione per adeguare un processo standard a caratteristiche aziendali\\
			-Di progetto: istanziato, usano risorse aziendali per raggiungere obbiettivi prefissati e con tempo limitato (progetti)\\
			
			\textbf{Processi specializzati/definiti}\\
			-Chiari, stabili, documentati\\
			-indipendenti dal modello di ciclo di vita adottato\\
			-Indipendenti dalle tecnologie\\
			-Indipendenti dal dominio applicativo\\
			-Indipendenti dalla documentazione richiesta\\
			
			\textbf{Processi di progetto}\\
			-Ben pianificati\\
			-Chiare scelte di specializzazione (definire lo scenario, le attività e i compiti aggiuntivi e specifici, organizzare le relazione tra i processi specializzati)\\
			-Massima attenzione nel condurre il progetto\\
			-Valutazione critica dell'esito (formalizzare le parti che operano bene)\\
			
			\textbf{Dipendono da:}\\
			-Dimensione del progetto\\
			-Complessità del progetto\\
			-Rischi identificati (da dominio applicativo e tecnologie)\\
			-Competenze ed esperienza delle risorse umane\\
			-Fattori dipendenti dal contratto\\
		\end{comment}
		
		\textbf{Organizzazione interna - Verifica}\\
		Ciclo PDCA:\\
		-Plan: definire attività, scadenze, responsabilità, risorse per raggiungere obbiettivi\\
		-Do: eseguire secondo i piani\\
		-Check: verificare l'esito delle azioni rispetto le attese\\
		-Act: applicare soluzioni correttive alle carenze\\
		
		\textbf{Processi e modelli di ciclo di vita}\\
		-La specifica dei processi non determina il modello di ciclo di vita\\
		-Il livello di coinvolgimento del cliente determina natura, funzione e sequenza dei processi di revisione\\
		-Quando il SW è parte di un sistema complesso il modello di ciclo di vita a \textit{livello di sistema} è spesso sequenziale.\\
		
		\textbf{Influenze sul modello di ciclo di vita}\\
		-Politiche di acquisizione e di sviluppo (versione unica o multipla, dipendenza da/verso altre componenti)\\
		-Natura, funzione e sequenza dei processi di revisione (interne, esterne, non bloccanti)\\
		-Necessità/utilità di fornire evidenze preliminari di fattibilità (prototipi bozza o baseline, studi e analisi preliminari)\\
		-Esigenza di iterazioni o di configurazioni (build, deployment)\\
			
			
			
			
	\clearpage
	\section{Ciclo di vita}
	
		\textbf{Stati principali}\\
		-Concezione\\
		-Sviluppo\\
		-Utilizzo\\
		-Ritiro\\
		
		\textbf{Organizzare le attività di processo}\\
		Si devono identificare dipendenze tra ingressi ed uscite, poi fissarle nel tempo assieme ai criteri di attivazione (pre-condizioni) e di completamento (post-condizioni)\\
		
		\textbf{Fase}\\
		Stazionamento in uno stato del ciclo di vita o in una transizione tra stati\\
		
		\textbf{Sistema di qualità}\\
		Associato al modello per assicurare conformità e maturità\\
		
		
		\textbf{Modello a cascata o sequenziale}\\
		Fasi:\\
		-Analisi (requisiti di sistema e software, etc)\\
		-Progettazione (Design, etc)\\
		-Realizzazione (Codifica, integrazione, collaudo, etc)\\
		-Manutenzione\\
		Eseguite in modo rigidamente sequenziale, no parallelismo, guidato da documentazione, codice solo alla fine, con pre-condizioni e post-condizioni per ogni fase\\
		Eccessiva rigidità, non permette modifiche ai requisiti, necessita di molta manutenzione, molto burocratico e poco realistico\\
		Big-gan integration: si integra tutto alla fine in un solo colpo, se non funziona difficile isolare e correggere il problema\\
		\textbf{Correzioni}\\
		-Prototipazione: usa e getta, scrivendo la documentazione si fanno delle prove\\
		-Cascata con ritorni: torno indietro per correggere/rifare una parte, rompendo il modello, iterazioni! - Modello iterativo\\
		
		\textbf{Modello iterativo}\\
		Applicabile a qualsiasi altro modello, consente l'adattamento (a evoluzione dei problemi, requisiti, soluzioni e tecnologie)\\
		Si ritorna indietro rispetto l'asse temporale\\
		
		\textbf{Modello incrementale}\\
		Fasi:\\
		-Define outline requiments\\
		-Assign requiments to increments (essenziale per poter procedere a incrementi)\\
		-Design system architecture (come le parti si compongono, essenziale per il parallelismo)\\
		finchè non ho il sistema finale:\\
		|-Develop system increment\\
		|-Validate increment\\
		|-Integrate increment\\
		|-Validate system\\
		Possibile svolgere gli incrementi in parallelo\\
		Riassumibile in : ``Analisi e progettazione'', poi ciclo su ``Progettazione di dettaglio'' e ``Implementazione dettaglio''\\
		
		\textbf{Modello evolutivo}\\
		Per uno scenario che varia (es Browser), molteplici versioni intermedie, ogni fase ammette iterazioni multiple e parallele\\
		Si basa su una analisi iniziale, poi cicla su analisi e progettazione ed sviluppo e validazione\\
		
		\textbf{Modello a componenti}\\
		Si basa sul riutilizzo di componenti\\
		Fasi:\\
		-Analisi requisiti\\
		-Analisi componenti\\
		-Adattamento dei requisiti (controllo cosa fa al caso mio e come dovrò modificarlo per soddisfare i requisiti)\\
		-Progettazione con riuso\\
		-Sviluppo e integrazione\\
		-Validazione di sistema\\
		
		\textbf{Modelli agili}\\
		-Niente regole rigide\\
		-Il software funzionante è più importante di una buona documentazione\\
		-Collaborare con il cliente, non negoziare\\
		-Essere reattivi, non mirare alla pianificazione\\
		Ma:\\
		-Adattare le regole è ok, ma bisogna mantenere un occhio su costi/benefici\\
		-La mancanza della documentazione fa lievitare il costo di manutenzione\\
		-Non pianificare significa non sapere se si sta avanzando e i rischi che si corrono\\
		\textbf{User story}\\
		Minuta, resoconto con il cliente, dialogando specifica i problemi e i requisiti, pezzo per pezzo\\
		Sarà una lista di cose che vuole, che preferirebbe e che non vuole, da usare per controllare l'avanzamento e l'efficacia\\
		\textbf{3 forme principali}\\
		La maggiore: \textbf{SCRUMB}\\
		Iterazione controllata, c'è un \textit{backlog} di cose da svolgere, si sceglie quali fare (\textit{sprint}) prendendo le più utili/necessarie/importanti, le faccio, le unisco in un incremento e itero nuovamente\\
		Sprint usualmente di circa 2 settimane, con misurazioni giornaliere brevi di tipo stand-up, intrusive!\\
		
		\textbf{Il ciclo di vita secondo SEMAT}\\
		Sequenza di punti/indicazioni suddivisi per categoria per aiutare a organizzare/misurare/controllare l'avanzamento e l'aver completato le principali problematiche durante tutto il ciclo di vita\\
		
		
	\clearpage
	\section{Gestione di progetto}
		\textbf{Fondamenti}\\
		Gestione di progetto - è un processo organizzativo per gestire altre attività\\
		-Processi di progetto istanziati da processi aziendali, a loro volta istanziati da standard di processo\\
		-Per stimare costi e le risorse necessarie\\
		-Per \textit{pianificare} attività ed assegnarle alle persone, in modo sistematico, disciplinato e quantificabile usando best practice\\
		-Controllare le attività e verificare i risultati per prendere provvedimenti\\
		
		\textbf{Funzione}\\
		Funzione aziendale, fissa, tra sviluppo, direzione (decisioni), amministrazione (gestione del supporto ai progetti), qualità (economicità)\\
		
		\textbf{Ruolo}\\
		Ruolo in un progetto, assegnato in base alla propria funzione\\
		
		\textbf{Ruolo: analista}\\
		Devono capire il problema e i requisiti/\textit{cosa fare}, pochi, competenze sul dominio del problema, grande influenza, presenti solo all'inizio\\
		
		\textbf{Ruolo: progettista}\\
		Deve capire \textit{come risolvere il problema}, attraverso la soluzione migliore come economicità, pochi, competenze sulle tecnologie, influenza sulle scelte tecniche e tecnologiche, a volte seguono il progetto fino alla manutenzione\\
		Devono anche fare l'analisi di fattibilità\\
		
		\textbf{Ruolo: programmatore}\\
		Molti, competenze tecniche, visione e responsabilità circoscritte, realizzano e mantengono il prodotto deciso dal progettista\\
		
		\textbf{Ruolo: responsabile}\\
		Aggrega i ruoli e li fa cooperare\\
		Responsabilità su pianificazione, gestione delle risorse umane, controllo e relazioni esterne\\
		Capacità tecniche necessarie per valutare rischi, scelte ed alternative\\
		
		\textbf{Ruolo: amministratore}\\
		Controllo ambiente di lavoro, amministrazione delle infrastrutture di supporto, risoluzione problemi riguardanti la gestione dei processi, gestione della documentazione, controllo di versioni e configurazione\\
		Funzione o ruolo nel progetto, dipende dalla organizzazione aziendale\\
		
		\textbf{Ruolo: verificatore}\\
		Gestiscono le verifiche, capacità di giudizio e relazione, competenze tecniche, esperienza professionale e conoscenza delle norme, sempre presenti\\
		
		\textbf{Ruolo: gestione qualità}\\
		Funzione aziendale, non ruolo, gestisce way of working aziendale\\
		Richiede applicazione rigorosa dei processi adottati, mantiene il ciclo PDCA\\
		
		\subsection{Responsabile}
		\textbf{Pianificazione di progetto}\\
			Con l'aiuto di strumenti, definizione delle attività per:\\
			-Pianificare lo svolgimento e controllarne l'attuazione\\
			-Avere una base per gestire l'allocazione delle risorse\\
			-Stimare e controllare scadenze e costi\\
			
			Svolgimento:\\
			-Identificazione della lista delle attività\\
			-Disposizione in ordine delle attività secondo le dipendenze\\
			-Stima delle risorse per attività\\
			-Allocazione del personale rispettando i vincoli (ore giornaliere, competenze, etc)\\
			-Creazione dei diagrammi del progetto, se qualcosa non va bene torno alla stima\\
			
			Realizzato con:\\
			-Diagrammi di Gantt\\
			-PERT\\
			-WBS\\
			
			\textbf{Gantt}\\
			Dislocazione temporale delle attività pianificate e eseguite, per controllare le stime con i progressi\\
			Utilizzabile anche con le persone per controllare sovrapposizioni o lavori in gruppo\\
			
			\textbf{PERT}\\
			Sottolinea dipendenze temporali tra le attività, per ragionare sulle scadenze, evidenzia il cammino critico (quello con slack minore o =0) e i vari slack (margine)\\
			
			\textbf{WBS}\\
			Struttura gerarchica delle attività, evidenzia le sotto-attività univocamente identificate, anche non sequenziali\\
			
			\textbf{Allocazione su più progetti}\\
			Risorse allocate in più processi per evitare sotto-utilizzo e richieste dei clienti, producono cammini critici\\
			
			\textbf{Stima costi di progetto}\\
			Definire durata in ore di lavoro e costo stimandolo secondo esperienza, analogia, competizione o algoritmo predittivo (non preciso), poi rapportandolo alle ore di calendario\\
			
			\textbf{Piano di progetto}\\
			Va documentato, si indica anche come si è giunti alla stima delle risorse necessarie\\
			Scritto dal responsabile, letto da verificatore e stakeholders, poi passato al team\\
			Contenuti:\\
			-risorse disponibili e le loro assegnazione alle attività\\
			-scansione delle attività nel tempo\\
			Obiettivi:\\
			-Organizzare le attività con efficienza\\
			-Facilitare la misurazione di avanzamento fissando \textit{milestone}\\
			Struttura tipica:\\
			-Introduzione (scopo e struttura)\\
			-Organizzazione del progetto\\
			-\textbf{Analisi dei rischi} - qualsiasi evento imprevisto fa modificare il piano di progetto, meglio prevedere\\
			-Risorse disponibili\\
			-Suddivisione del lavoro\\
			-Calendario delle attività\\
			-Meccanismi di controllo e rendicontazione\\
			
			\textbf{Rischi}\\
			-Sforare i tempi/budget\\
			-Risultati insoddisfacenti\\
			Motivi:\\
			-Tecnologie di lavoro\\
			-Rapporti interpersonali\\
			-Organizzazione del lavoro\\
			-Requisiti e rapporti con gli stakeholder\\
			-Tempi e costi\\
		
			\textbf{Gestione dei rischi}\\
			Durante la pianificazione, sempre sotto gestione del progetto\\
			-Identificazione (in qualsiasi ambito, di qualsiasi tipo)\\
			-Analisi (probabilità che accadano e il livello di impatto)\\
			-Pianificazione (come mitigare o evitare)\\
			-Controllo (durante tutto lo svolgimento, con misurazioni, per raffinare le strategie e modificare la lista dei rischi identificati)\\
			Pianificando su vincoli lunghi ma in periodi brevi ho errori minori\\
			
			\textbf{Baseline}\\
			Prodotto prototipale, è il risultato di avanzamenti misurabili\\
			
			\textbf{Milestone}\\
			Concretizzata da almeno una baseline, punto nel tempo strategico e di riferimento, meta da raggiungere con dei risultati certi/solidi che non deve essere ritoccata (incrementali)\\
			Requisiti per buoni milestone:\\
			-specifiche per obbiettivi\\
			-delimitate per ampiezza ed ambizioni, raggiungibili\\
			-incrementali e misurabili come impegno necessario\\
			-coerenti con la strategia di progetto\\
			-traducibili in compiti assegnabili\\
			-puntuali\\
			-dimostrabili agli stakeholder\\
			
			\textbf{Tempo persona}\\
			Diverso dal tempo di calendario, influenzato da efficacia ed efficenza, difficile da valutare\\
		
	\clearpage
	\section{Amministrazione} 
		\textbf{Amministrazione di sistema}\\
		Equipaggiare, organizzare e gestire l'ambiente di lavoro e di produzione, a supporto dei processi istanziati dai processi, scelte tecnologiche concordate, no scelte gestionali\\
		-Reperimento, gestione, organizzazione e manutenzione di risorse informatiche e di servizi\\
		-Gestione del controllo di versione\\
		-Gestione della configurazione, del build e dei test e validazioni automatici\\
		-Gestione dei documenti\\
		-Gestione dell'ambiente di lavoro\\
		-Redazione e manutenzione di regole e procedure di lavoro - \textit{norme}\\
		
		\textbf{Issues o ticket}\\
		Idea, questione, problema, attività, etc, considerabile in due modi:\\
		-in avanti: compito, c'è una attività da fare, che posso scegliere e svolgerla, oppure che qualcuno sceglie per me pianificando, in base a un ordine di priorità\\
		-all'indietro: cosa da fare o considerare, qualcuno deve gestirlo\\
		
	\clearpage
	\section{Analisi dei requisiti}
		\textbf{Requisiti secondo IEEE}\\
		1 - Condizione (capability) da chi la offre - capacità di risolvere un problema o raggiungere un obbiettivo\\
		2 - Condizione (capability) da chi la richiede - che deve essere soddisfatta o posseduta da un sistema per aderire a un obbligo (contratto, standard, specifica, documento formale)\\
		3 - Descrizione documentata di una condizione come in 1 o 2\\
		
		\textbf{Qualifica}\\
		Verifica + Validazione\\
		\textit{Verifica}: processo di supporto, accertamento che l'esecuzione delle attività non abbia introdotto errori, rivolto ai processi, da fare per OGNI componente\\
		Per la verifica serve piano di verifica che si basa sul way of working\\
		\textit{Validazione}: controllo rivolto SOLO al prodotto finale, lungo e costoso, accertarsi che il prodotto realizzato corrisponda alle attese\\
		
		
		
		
		
		\textbf{Analisi}\\
		-Studio dei bisogni e delle fonti del dominio applicativo\\
		-Prima classificazione dei requisiti\\
		-Modellazione concettuale del sistema - secondo gli use case\\
		-Assegnazione dei requisiti a parti distinte del sistema - secondo gli use case\\
		-Negoziazione con il committente, consolidamento della classificazione dei requisiti (l'ordine di importanza)\\
		
		\textbf{Piano di qualifica}\\
		-Definizione delle strategie di verifica\\
		-Metodi, tecniche e procedure da usare per la validazione\\
	
		\textbf{Attività di analisi}\\
		-Studiare e definire il problema\\
		|-Identificare il prodotto da commissionare (compito del cliente)\\
		|-Capire cosa realizzare (cliente + fornitore)\\
		|-Definire gli accordi contrattuali (cliente + fornitore)\\
		-Verificare le implicazioni di costo e di qualità\\
		|-Requisiti espliciti o impliciti, diretti o derivati per la soddisfazione del cliente\\
		-Studio dei bisogni e delle fonti (\textit{identificare}, \textit{specificare} e \textit{classificare} i requisiti dal punto di vista committente e conoscendo l'ambito)\\
		|-identificare: precisamente, che requisito serve\\
		|-specificare: secondo quali limitazioni/regole\\
		|-classificare: capirne l'importanza o la negoziabilità\\
		-Modellazione concettuale del sistema\\
		|-Partizionamento in componenti per l'allocazione dei requisiti, con diagramma dei casi d'uso (analisi - cosa, non il come)\\
		-Ripartizione dei requisiti a parti del sistema\\
		%-----------------
		-Accertarsi della soddisfacibilità dei requisiti\\
		-Assicurarsi che i requisiti concordati siano solo e tutti quelli necessari e sufficienti\\
		-Determinare con il cliente l'utilità strategica\\
		%-----------------
		
		\textbf{Documentazione - processo di supporto}\\
		-Definizione dei bisogni (utente - contrattuali, il cosa e SW - il come)\\
		Analisi di fattibilità - del fornitore, riservato\\
		Analisi dei requisiti - documento contrattuale\\
		
		\textbf{Gestione del prodotto - processo di supporto}\\
		-Tracciamento requisiti (sapere da che esigenza arrivano)\\
		-Impostazione e configurazione della configurazione - con versioning, automatizzata\\
		-Gestione dei cambiamenti (discuterli, capirli e motivarli, sempre con delle regole)\\
		\includegraphics[width=10cm]{verifica.jpg}\\
		\includegraphics[width=10cm]{analisi_progettazione.jpg}\\
		
		%?---------------------------------------------------------------------??
		%?---------------------------------------------------------------------??
		%?---------------------------------------------------------------------??
		%?---------------------------------------------------------------------??
		\textbf{Approccio funzionale}\\
		-Studio di fattibilità porta ad analisi dei requisiti in linguaggio naturale + supporto di linguaggi formali o semi-formali (diagramma dei casi d'uso)\\
		-Specifica tecnica in linguaggi formali, definizione di funzione e profilo operazionale\\
		-Top-down - programmazione procedurale\\
		
		\textbf{Approccio object-oriented}\\
		-Studio di fattibilità porta ad analisi dei requisiti in formalismi grafici (diagramma dei casi d'uso)\\
		-Continuità logica con la progettazione mediante le classi\\
		-Bottom-up, aggregazione di parti, design pattern e riutilizzo, programmazione ad oggetti\\
		
		%\textbf{Approccio con modello agile}\\
		%???
		
		\textbf{Studio di fattibilità}\\
		-Valutare rischi, costi e benefici dal punto di vista di fornitore e cliente\\
		-Decidere se procedere o meno con le conoscenze disponibili o con un piano di formazione sostenibile\\
		
		\textbf{Fattibilità tecnico-organizzativa}\\
		Disponibilità di tecnologie, soluzioni algoritmiche e architetturali possibili, piattaforme idonee per l'esecuzione\\
		-Rapporto costi benefici:\\
		Confrontare il mercato attuale e quello futuro, valutare costo e redditività\\
		-Individuare rischi (complessità e incertezze)\\
		-Valutazione delle scadenze temporali - disponibilità delle risorse necessarie\\
		-Valutare alternative:\\
		|-Scelte architetturali come sistema decentralizzato, client-server, etc)\\
		|-Strategie realizzative: riuso o sviluppo da zero\\
		|-Strategie operative: Avvio, esercizio, manutenzione del sistema e formazione utenti\\
		
		
		\textbf{Classificazione dei requisiti}\\
		Mettere ordine nei requisiti facilita la comprensione, manutenzione e tracciamento\\
		Attributi del prodotto:\\
		-Caratteristiche richieste al sistema, cosa devo fare? requisiti funzionali, prestazionali e di qualità\\
		-Vincoli sui processi impiegati, come defo farlo? requisiti di vincolo realizzativo, normativo o contrattuale\\
		I requisiti devono essere verificabili:\\
		-requisiti funzionali: test, dimostrazione formale o revisione\\
		-requisiti prestazionali: misurazione\\
		-requisiti qualitativi: verifica ad hoc\\
		-requisiti dichiarativi (vincoli): revisione\\
				
		\textbf{Utilità strategica dei requisiti}\\
		-Obbligatori: irrinunciabili\\
		-Desiderabili: con valore aggiuntivo riconoscibile\\
		-Opzionali: relativamente utili o contrattabili più avanti\\
		
		\textbf{Specifica del progetto - secondo IEEE 830-1998}\\
		Deve essere:\\
		-Priva di ambiguità\\
		-Corretta\\
		-Completa\\
		-Verificabile\\
		-Consistente\\
		-Modificabile\\
		-Tracciabile\\
		-Ordinata per rilevanza\\
		
		Parti:\\
		-Introduzione (scopo documento e del progetto, glossario, riferimenti normativi e non, struttura del documento)\\
		-Descrizione generale (prospettive, funzioni del prodotto, caratteristiche degli utenti, vincoli, assunzioni e dipendenze)\\
		-Specifica requisiti (definizione requisiti utente e di sistema, prima composizione del sistema, evoluzione attesa del sistema)\\
		-Eventuali appendici\\
		
		\textbf{Verifica dei requisiti}\\
		Eseguita su un documento organizzato, tramite:\\
		-Walkthrought: lettura a largo spettro\\
		-Ispezione: lettura mirata e strutturata\\
		Matrice delle dipendente (necessità e sufficienze) per tracciare\\
		Ricercando chiarezza espressiva, strutturale (separare requisiti funzionali e non) ed atomicità e aggregazione (requisiti elementari, correlazioni chiare)\\
		
		Identificazione e classificazione, con ID, numerazione di sequenza o coppie <categoria, numero>\\
		Gestire i cambiamenti valutandone l'impatto e la fattibilità tecnica\\
		Necessità di tracciare\\
		
		\textbf{Riuso}\\
		Progettazione influenzabile da esigenza o opportunità di riuso di:\\
		-componenti aziendali preesistenti\\
		-componenti commerciali\\
		.componenti imposte dal cliente\\
		
		\textbf{Stati di progresso per SEMAT} 
		-Conceived: il committente è identificato e gli stakeholder vedono una opportunità per il progetto\\
		-Bounded: I bisogni macro sono chiari, meccanismi di gestione dei requisiti fissati (configurazione e cambiamento)\\
		-Coherent: I requisiti sono classificati, quelli essenziali sono chiari e ben definiti\\
		-Acceptable: I requisiti fissati definiscono un sistema soddisfacente per gli stakeholder\\
		-Addressed: Il prodotto soddisfa i principali requisiti, possibile il rilascio e l'uso\\
		-Fulfilled: Il prodotto soddisfa abbastanza requisiti da avere la piena approvazione degli stakeholder\\
		
		
		
\end{document}
